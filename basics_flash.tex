% !TeX root = main.tex
% !TeX spellcheck = de_DE
% !TeX encoding = utf8

\chapter{Flash Memory}
Flash memory often in the form of solid state drives (SSD) are becoming more and more of primary storage for computers. The cost per storage has been decreasing steadily therefore driving adoption. Especially the low access latency and the high possible throughput compared so spinning disk hard drives are an advantage. In the application predominantly NAND type flash is used due to its higher density. 


\section{Flash Basics}
NAND flash which will be discussed is of most interest for SSDs due to its high density and therefore lower cost per bit. \cref{fg_tans} shows a floating gate transistor which is used to store data. By inserting charge into the floating gate it is possible to change the threshold voltage of the transistor. 

\begin{figure}
    {make image of floating gate flash transistor}
    \centering
    \caption{A floating gate field effect transistor}
    \label{fg_tans}
\end{figure}

\begin{figure}
    {make image of nand string}
    \centering
    \caption{Multiple transistors are arranged to make a NAND string}
\end{figure}

\section{Error Types}

\cite{ZaTu16}