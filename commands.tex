% !TeX root = main.tex
% !TeX spellcheck = de_DE
% !TeX encoding = utf8


\usepackage[T1]{fontenc}
\usepackage[utf8]{inputenc}
\usepackage[english]{babel}
\usepackage{bm}

\usepackage[bookmarks,		%Lesezeichen für Reader erzeugen
	pageanchor,		%Inhaltsverzeichnis mit Links
	colorlinks,		%farbige Links, statt Rahmen
	citecolor=gray,
	linkcolor=black,
	urlcolor=black,
	filecolor=black,
	menucolor=black,
	pdftitle={Master Thesis},
	pdfauthor={Henry Bathke},
	pdfsubject={Master Thesis: Design and Implementation of an LDPC-based FEC encoder/decoder suitable for Storage devices},
	]{hyperref}
\usepackage{amsmath}
\usepackage[english,nameinlink,noabbrev]{cleveref}
\usepackage{amssymb}
\usepackage{gensymb}
\usepackage{pdfpages}
\usepackage[backend=bibtex]{biblatex}

%broken biblatex http://tex.stackexchange.com/questions/311426/bibliography-error-use-of-blxbblverbaddi-doesnt-match-its-definition-ve
\makeatletter
\def\blx@maxline{77}
\makeatother

\usepackage{multirow}
\usepackage[
	%disable
	]{todonotes}
\usepackage[tikz]{bclogo}
\usepackage[binary-units=true]{siunitx}
\usepackage{varwidth}
\sisetup{locale = DE}
\usepackage{xcolor}
\usepackage{geometry}
\usepackage{etoolbox}
\usepackage{filecontents}
\usepackage{booktabs}
\usepackage{subcaption}
\usepackage[framed,numbered,autolinebreaks,useliterate]{mcode}
\usepackage{newunicodechar}
\usepackage{lastpage}
\usepackage{import}
%\newunicodechar{ß}{{\ss{}}}
\usepackage[normalem]{ulem}
\usepackage{rubfonts2009}
\usepackage[autostyle=true]{csquotes}
%\usepackage{tikz}
%\usepackage[europeanresistors]{circuitikz}
\usepackage{pgfplots}
\usepackage{pgfplotstable}
\usetikzlibrary{shapes, arrows.meta, positioning, decorations.pathmorphing, math, calc, fit, chains, matrix, decorations.pathreplacing}
\sisetup{per-mode=symbol} 

\tikzset{
	block/.style={draw, rectangle, minimum width = 2cm, minimum height = 1.4 cm, align=center},
	arrow/.style={-{Stealth[scale=1.5]}}
}

\newcommand{\editRem}[1]{\ifmmode\hbox{\sout{$#1$}}\else\sout{#1}\fi}
\newcommand{\editAdd}[1]{\textcolor{green!50!black}{#1}}
\newcommand{\editRep}[2]{\editRem{#1} \editAdd{#2}}
\newcommand{\editCom}[1]{\textcolor{gray}{(#1)}}
\bibliography{database.bib}

\newcommand{\executeiffilenewer}[3]{%
	\ifnum\pdfstrcmp{\pdffilemoddate{#1}}%
	{\pdffilemoddate{#2}}>0%
	{\immediate\write18{#3}}\fi%
}
% set inkscape binary path according to operating-system
\newcommand{\Inkscape}{inkscape }%

% includesvg[scale]{file} command
\newcommand{\includesvg}[2][1]{%
	\executeiffilenewer{#2.svg}{#2.pdf}{%
		\Inkscape  -z -D --file="#2.svg" --export-pdf="#2.pdf" --export-latex}%
	\scalebox{#1}{\import{}{#2.pdf_tex}}%
}


\pgfplotsset{compat=1.6}

\global\mdfdefinestyle{taskstyle}{%
	linecolor = black!20!white, 
	linewidth=3pt,%
	leftmargin=0.2cm,
	topline=false, 
	bottomline=false, 
	rightline=false
}


\newcounter{taskcount}[subsection]                                              
\newenvironment{task}
{
\refstepcounter{taskcount}
%\begin{bclogo}[couleur = blue!20, couleurBord=black, logo=\bccrayon]{Aufgabe \alph{taskcount})}
%\begin{leftbar}
\begin{mdframed}[style=taskstyle]
	{
		\large
		\textbf{Aufgabe \alph{taskcount})}
	}
	\vspace{0.5em}
	\newline
}{
\end{mdframed}
%\end{leftbar}
%\end{bclogo}
}
	
\pgfplotscreateplotcyclelist{hollow}{%
	mark=o,color=red,thick\\%
	mark=square,color=brown,thick\\%
	mark=x,color=green!50!black,thick\\%
	mark=triangle,color=blue,thick\\%
	mark=diamond,color=black,thick\\%
}	

%\renewcommand*\sectionformat{}
\DeclareMathOperator{\sign}{sign}
\DeclareMathOperator*{\argmin}{arg\,min}


\usepackage[headsepline, headtopline, plainheadsepline, plainheadtopline, footsepline, plainfootsepline]{scrlayer-scrpage}
%\ihead*{Henry Bathke}
\chead*{}
\automark[chapter]{chapter}
\ohead*{\headmark}
\cfoot*{\thepage/\pageref{LastPage}}
\pagestyle{scrheadings}

\lstdefinelanguage{VHDL}{
   morekeywords={
     library,use,all,entity,is,port,in,out,end,architecture,of,
     begin,and
   },
   morecomment=[l]--
}

\colorlet{keyword}{blue!100!black!80}
\colorlet{comment}{green!90!black!90}
\lstdefinestyle{vhdl}{
   language     = VHDL,
   basicstyle   = \small\ttfamily,
   tabsize		= 4,
   keywordstyle = \color{keyword}\bfseries,
   commentstyle = \color{comment}
}