\documentclass[border=1cm]{standalone}

%\usepackage{tikz}
\usepackage{amsmath}
\usepackage{pgfplots}
\usepackage{gensymb}
\usepackage{pgfplots}
\usepackage{pgfplotstable}
\usepackage{bm}
\usepackage{amssymb}
\usepackage[siunitx]{circuitikz}
\usetikzlibrary{shapes, arrows.meta, positioning, decorations.pathmorphing, math, calc, fit, chains, matrix, decorations.pathreplacing, decorations.markings}

\pgfplotsset{compat=1.15}
\newcommand{\mosfet}[1]{%
    \draw (#1.B) -- ($(#1.B)+(0.3,0)$);%
    \draw ($(#1.B)+(0.3,0.3)$) -- ($(#1.B)+(0.3,-0.3)$);%
    \draw ($(#1.B)+(0.4,0.3)$) -- ($(#1.B)+(0.4,-0.3)$);%
    \draw ($(#1.B)+(0.5,0.4)$) -- ($(#1.B)+(0.5,-0.4)$);%
    \draw ($(#1.B)+(0.5,0.3)$) -| (#1.D);
    \draw ($(#1.B)+(0.5,-0.3)$) -| (#1.S);
}

\begin{document}
\begin{tikzpicture}
    \begin{semilogyaxis}[width=11cm, xlabel={$E_b / N_0$} in dB, ylabel={Error Rate}, legend pos=south west]
        \pgfplotstableread{data/sw_norm_params.dat}\paramopt
        \foreach \x in {1,...,11} {
            \addplot table [col sep=tab,x index={0}, y index={\x}] {\paramopt};
            \pgfplotstablegetcolumnnamebyindex{\x}\of{\paramopt}\to{\colname}
            \addlegendentryexpanded{Normalization \colname}
        }
    \end{semilogyaxis}
\end{tikzpicture}
    
\end{document}
